%%%%%%%%%%%%%%%%%%%%%%%%%%%%%%%%%%%%%%%%%
% Twenty Seconds Resume/CV
% LaTeX Template
% Version 1.1 (8/1/17)
%
% This template has been downloaded from:
% http://www.LaTeXTemplates.com
%
% Original author:
% Carmine Spagnuolo (cspagnuolo@unisa.it) with major modifications by 
% Vel (vel@LaTeXTemplates.com)
%
% License:
% The MIT License (see included LICENSE file)
%
%%%%%%%%%%%%%%%%%%%%%%%%%%%%%%%%%%%%%%%%%

%----------------------------------------------------------------------------------------
%	PACKAGES AND OTHER DOCUMENT CONFIGURATIONS
%----------------------------------------------------------------------------------------

\documentclass[letterpaper]{twentysecondcv} % a4paper for A4

%----------------------------------------------------------------------------------------
%	 PERSONAL INFORMATION
%----------------------------------------------------------------------------------------

% If you don't need one or more of the below, just remove the content leaving the command, e.g. \cvnumberphone{}



\cvname{Felipe Rocha} % Your name
\cvjobtitle{DSc. Comp. Modelling} % Job title/career

\cvdate{17 January 1990, Natal, Brazil} % Date of birth
\cvaddress{f.rocha.felipe@gmail.com \newline felipe.figueredorocha@epfl.ch} % Short address/location, use \newline if more than 1 line is required
\cvnumberphone{+41 77 8158008} % Phone number
\cvsite{Orcid: 0000-0001-6893-1109 \newline lattes.cnpq.br/8844184831413061}
\cvlinks{Github: github.com/felipefr \newline sites.google.com/view/feliperocha} 


%----------------------------------------------------------------------------------------

\begin{document}

%----------------------------------------------------------------------------------------
%	 Education
%----------------------------------------------------------------------------------------
\section{Introduction} \\
I am currently postdoctoral fellow in the Applied Mathematics department at Ecole Polytechnique Fédéral de Lausanne since Dec'2019, working mainly with Reduced Models and Machine Learning applied for numerical solution of PDEs, under supervision of Prof. Annalisa Buffa and Prof. Simone Deparis. I earned my PhD and Master in Computing Modelling at National Laboratory for Scientific Computing (LNCC), Brazil, both under supervision of Prof. Pablo Blanco and Prof. Raúl Feijóo. In this period, I was also visitor student at the Zienkiewicz Centre for Computational Engineering, Swansea, United Kingdom, working with Prof. Eduardo de Souza Neto. My Bachelor background is in Mechanical Engineering, earned at Universidade Federal do Rio Grande do Norte (UFRN), Brazil, with an exchange period at École d'Arts et Métiers Paristech, France. My other main research interests are: constitutive modelling of soft tissues with emphasis in softening and failure mechanisms; computational homogenisation and multi-scale modelling; continuum mechanics with emphasis in nonlinear solid mechanics; numerical methods with emphasis in Finite Element Method.

\Education{
2019 | DSc. Computational Modelling LNCC, Petropolis, Brazil | GPA:3.83/4 \vskip 0.1cm
2014| MSc. Computational Modelling LNCC, Petropolis, Brazil | GPA:3.9/4 \vskip 0.1cm
2012 | B.E. Mechanical Engineering UFRN, Natal, Brazil | GPA:8.8457/10} % To have no Education section, just remove all the text and leave \Education{}

\ExtraCurricular{
Scientific Computing: Numerical Methods; Numerical Solution of Differential Equations; Numerical Analysis of Finite Element Methods;Optimisation;
\vskip 0.1cm 
Mathematics: Real Analysis; Functional Analysis; Variational Calculus; Probability and Statistics.
\vskip 0.1cm
Computational Mechanics: Continuum Mechanics; Computational Transport Phenomena.
\vskip 0.1cm
Computer Science: Data Structures ; Software Design ; POO; Parallel Processing. 
}
%----------------------------------------------------------------------------------------
%	 SKILLS
%----------------------------------------------------------------------------------------

% Skill bar section, each skill must have a value between 0 an 6 (float)
\skills{B - Basic, F- Fluent \newline
Languages: Python (F), C/C++ (F), Fortran 77/90 (F), Matlab (B).
\newline Extensions and Libraries : Tensorflow/Keras (F), Numpy/Scipy (F), Fenics (F), OpenMP (B), MPI/mpi4py (B), Petsc (B).  
\newline  Other: Linux/Mac (F), Latex (F), Gmsh (F), Git (B), Bash (B). }

\linguas{
Portuguese (Native), English (C1, IELTS: 6.5, TOEFL: 89), French (B2), Italian (B1), Spanish (B1), German (A2).
}


\makeprofile % Print the sidebar


\section{Research Dissertations}

\begin{twenty}
	\twentyitem{Thesis}{(Jan'15-Apr'19) Multiscale Modelling of Fibrous Materials: from the elastic regime to failure detection in soft tissues}{}{Supervisors: Prof. Pablo Javier Bland and Prof. Raul Feijoo. \newline
	Institution: National Laboratory for Scientific Computing (LNCC), Petropolis, Brazil.
	}
	\twentyitem{Master}{(Mar'13-Dec'14) Basics Aspects of Multi-Scale Modelling of Biological Tissues [In Portuguese]}{}{Supervisors and Institution : as above.}
	\twentyitem{Final Project}{(Aug'12-Dec'12) Development of a Computational Dynamics Software for Multiple Rigid Bodies Analysis [In Portuguese]}{}{Supervisors: Prof. Wallace Bessa. \newline
	Institution: Federal University of Rio Grande do Norte (UFRN), Natal, Brazil.}
	\twentyitem{Final Project}{(Dec'11-Jul'12) Modelling and Numerical Simulation of the Machining Process in an Automobile Part [In French]}{}{Supervisors: Prof. Philippe Lorong and Prof. Jerôme Duchemin. \newline
	Institution: Arts et Métiers Paristech (ENSAM), Paris, France.}
	%\twentyitemshort{<dates>}{<title/description>}
\end{twenty}

\section{Short- and Long-term Academic Visits}

\begin{twenty}
\twentyitem{Sep-Out'19}{Prof. Sidarta Lima, Federal University of Rio Grande do Norte, Natal, Brazil}{}{I gave a short course on Mixed Formulations to Darcy's Equation to his group of graduate students and expanded their in house solver to Raviart-Thomas element. I was funded by Petrobras during this time.}
\twentyitem{Sep-Fev'18}{Prof. Eduardo de Souza Neto, Swansea University, Swansea, UK}{}{I visited the Zienkiewicz Centre for Computational Engineering (ZCCE) as part of the sandwich exchange program during my PhD. During this period, under supervision of Prof. de Souza Neto, expert in the domain of Computational Plasticity and Constitutive Multiscale theories, we developed and implemented a nouvel micromechanical inelastic model for networks of fibres. I was funded by PSDE/CAPES.}

\twentyitem{Apr'17}{Prof. Anne Robertson, University of Pittsburgh, Pittsburgh, USA} {}{I spent 20 days observing mechanical experiments in livings tissues (arteries) and collaborating with microscopic modelling of collagen tissues.}

\twentyitem{Jun-Jul'15}{Prof. Pablo Sanchez and Prof. Alfredo Huespe, Centro de Investigación de Métodos Computacionales (CIMEC), Santa Fe, Argentina}{}{As part of my thesis, I spent a month collaborating and learning from two experts in the domain of computational fracture mechanics field.}

\twentyitem{Aug-Jul'12}{Arts et Métiers Paristech (ENSAM), Paris, France}{}{I took part of undergraduate studies in a sandwich exchange program in France. I was enrolled in the third (last) year, which consisted in a  masters degree in ''Prototypage Virtuel'' (Computational Solid Mechanics) including a final research project. (Funded by  Brafitec/CAPES).}
\end{twenty}

\makeprofile % Print the sidebar
\section{Publications}
\item \textbf{Peer-reviewed journal:}
\begin{enumerate}
\item Felipe Figueredo Rocha, Pablo Javier Blanco, Pablo Javier Sánchez, and Raúl Antonino Feijóo. Multi-scale modelling of arterial tissue: Linking networks of fibres to continua. \textit{Computer Methods in Applied Mechanics and Engineering}, 341:740–787, 2018
\item Felipe Figueredo Rocha, Pablo Javier Blanco, Pablo Javier Sánchez, Eduardo de Souza Neto, and Raúl Antonino Feijóo. Damage-driven strain localisation in networks of fibres: A computational homogenisation approach (minor revision resubmitted). \textit{Computer and Structures}, 2021.
\item Pablo Javier Blanco, Pablo Javier Sánchez, Felipe Figueredo Rocha, Sebastian Toro, and Raúl Antonino Feijóo. A consistent multiscale mechanical formulation for media with randomly distributed voids (in preparation). \textit{Computer Methods in Applied Mechanics and Engineering}, 2021.
\item Felipe Figueredo Rocha, Simone Deparis, Pablo Antolin, Annalisa Buffa. DeepBoundary : A Machine Learning Approach to Enhance Multi-scale  Solid Mechanics (in preparation). \textit{Journal of Computational Physics}, 2021.

\end{enumerate}	
\vskip -0.1cm
\item \textbf{Complete articles in conference proceedings:} 
\begin{enumerate}
	\item F.F. Rocha, P.J. Blanco, R.A. Feijóo, P.J. Sanchez, and A.E. Huespe. A multi-scale approach to model arterial tissue. \textit{In Ibero-Latin American Congress on Computational Methods in Engineering} (CILAMCE), Rio
	de Janeiro, 2015.
\end{enumerate}	
\vskip -0.1cm
\item \textbf{Extended Abstracts in Conferences:}
\begin{itemize}
\item F.F. Rocha; P.J. Blanco ; P.J. Sánchez; R.A. Feijóo. On the constitutive modeling for fibrous tissues. In: International Conference on Computational and Mathematical Biomedical Engineering, 2017, PITTSBURGH. International Conference on Computational and Mathematical Biomedical Engineering Proceedings, 2017.
\end{itemize}
\item \textbf{Abstracts in Conferences:} 
\begin{itemize}
    \item F.F. Rocha; P.J. Blanco; de Souza Neto, E.; P.J. Sánchez, R.A. Feijóo. An computational homogenisation approach to assess the strain localisation due to damage in fibre networks. XVI International Conference on Computational Plasticity. Fundamentals and Applications, Barcelona, Spain, 2021, CIMNE (To be held).
	\item P.J. Blanco, P.J. Sánchez, F.F. Rocha, Toro, S.; R.A. Feijóo.
	Multiscale formulation for materials with randomly distributed voids: minimally constrained and more restrictive multiscale sub-models. In: XII Argentine Congress on
	Computational Mechanics, 2018, San Miguel de Tucumán.
	Mecánica Computacional. Santa Fé: Asociación Argentina de Mecánica Computacional, 2018. v.XXXVI.
	p.1683 - 1683
	\item F.F. Rocha; P.J. Blanco; de Souza Neto, E.; P.J. Sánchez, R.A. Feijóo.
	Towards post-critical multiscale modelling of damage in biological fibrous tissues.
	In: XII Argentine Congress on Computational Mechanics, 2018, San Miguel de Tucumán.
	Mecánica Computacional. Santa Fé: Asociación Argentina de Mecánica Computacional, 2018. v.XXXVI.
	p.1875 - 1875
	\item F.F. Rocha, P.J. Blanco, P.J. Sánchez, R.A. Feijóo.
	A Multiscale Approach to Study Softening Mechanisms in Arterial Tissue In: EMI2017-IC - 2017 EMI
	International Conference, 2017, Rio de Janeiro. EMI2017-IC - 2017 EMI International Conference Proceedings. , 2017.
	\item Toro, S., F.F. Rocha, P.J. Sánchez, P.J. Blanco, A.E. Huespe, R.A. Feijóo.
	Modelado Multiescala de Materiales: Análisis de Condiciones de Borde en Micro-Estructuras con Poros y/o
	Inclusiones que Alcanzan la Frontera del RVE In: Congreso sobre Métodos Numéricos y sus Aplicaciones,
	2017, La Plata. Anais do ENIEF 2017. La Plata: Asociación Argentina de Mecánica Computacional, 2017. v.XXXV. p.1309 -1309.
\end{itemize}


\newpage

\makeprofile

\section{Other Attended Conferences}
\begin{itemize} 
 \item \textbf{(8-10 Jan'2020)} International Workshop on Scientific Machine Learning, University of Cologne, Germany.
\item \textbf{MSML2020, (20-24 Jul'2020)}  Mathematical & Scientific Machine Learning Conference,
 Online, Princeton University (Online Event).
\item \textbf{mathml2020, (3-7 Aug'2020)}  LMS-Bath Symposium on the Mathematics of Machine Learning, Online, University of Bath (Online Event).
\end{itemize}

\section{Invited Talks}
\begin{itemize}
\item \textbf{EAMC 2021 (2021), LNCC} Galerkin convida Mr. Deep para um café (in Portuguese).
\item \textbf{Postgraduation Seminar (2021), LNCC} Aprendizado de Máquina em Computação Científica com Aplicações à Solução Numérica de EDPs (in Portuguese).
\end{itemize}

\section{Relevant Complementary Education}
\item \textbf{2020 (16h, listener)} (MATH-631) Mathematical foundations of neural networks, EPFL, Lausanne, Switzerland.
\item \textbf{2019 (36h, listener)} Introdução ao Aprendizado de Máquina, LNCC, Petrpolis, Brazil.
\item \textbf{2018 (4h)} Python for HPC, LNCC, Petropolis, Brazil.
\item \textbf{2015 (6h) } New Formulations of Finite Element Method, LNCC, Petropolis, Brazil.
\item \textbf{2014 (32h) } Biomech. Summerschool: Trends of Modelling, TUGraz, Graz, Austria.
\item \textbf{2014 (4h) } Topological Asymptotic Analysis, LNCC, Petropolis-RJ, Brazil. 
\item \textbf{2014 (7h) } Object-Oriented Finite Element Method, LNCC, Petropolis-RJ, Brazil.

\section{Teaching Experiences}
\item \textbf{Spring 2021: Analysis II (EPFL):} Teaching assistant.
\item \textbf{Spring 2020: Numerical Analysis (EPFL):} Teaching assistant.
\item \textbf{Jun'-Sep'2018: Numerical Methods (LNCC):} I worked as a tutor for students pursuing MS and PhD degrees in Computational Modelling at LNCC.
\item \textbf{Mar'-May'2018: Introduction to Modelling:} See comments above.
\item \textbf{Mar'-Oct'2017: Preparation for the Brazilian Mathematics Olympiads:} I worked as a online tutor for distinguished high-school students of public schools.

\section{Supervision} \\
\textbf{2020-actual, Co-supervisor, PhD Thesis} Ronaldo Dias dos Santos Junior, Computational and Mathematical Modelling of Polymer Injection in Oil Resevoirs (in preparation), UFRN, Natal-RN, Brazil.  

\section{Reviewer} \\
\textbf{2021} National Congress of Applied and Computational Mathematics, CNMAC 2021 (Online Event), Brasil. \\
\textbf{2021}  Academic Meeting on Comp. Modelling, EAMC 2021, LNCC, Petropolis, Brazil. \\
\textbf{2020-actual} Engineering Computations.  

\section{Participation in Examination Boards} \\
\textbf{2020, Bachelor's Final Project} Henrique Ribeiro da Silva, Dinâmica dos Fluidos Computacional: Uma aproximação paralelizada via Smoothed Particle Hydrodynamics, CEFET, Petropolis-RJ, Brazil.

\section{Achievements}
\item \textbf{2017 - ''Aluno nota 10'' (LNCC):} scholarship awarded to the best PhD student.
\item \textbf{2014 - ''Aluno nota 10'' (LNCC):} scholarship awarded to the two best MS students.}
\item \textbf{2012 - Summa cum Laude (UFRN): } higher GPA of its undergraduate class.}
\item \textbf{2006 - Silver Medal:} on the Brazilian Mathematics Olympiads of Public Schools.




% \section{Other information}

%\subsection{Review}

%Alice approaches Wonderland as an anthropologist, but maintains a strong sense of noblesse oblige that comes with her class status. She has confidence in her social position, education, and the Victorian virtue of good manners. Alice has a feeling of entitlement, particularly when comparing herself to Mabel, whom she declares has a ``poky little house," and no toys. Additionally, she flaunts her limited information base with anyone who will listen and becomes increasingly obsessed with the importance of good manners as she deals with the rude creatures of Wonderland. Alice maintains a superior attitude and behaves with solicitous indulgence toward those she believes are less privileged.
%----------------------------------------------------------------------------------------

\end{document} 
