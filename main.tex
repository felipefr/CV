\documentclass[french]{article}
\usepackage[T1]{fontenc}
\usepackage[utf8]{inputenc}
\usepackage{lmodern}
\usepackage[a4paper]{geometry}
\usepackage{babel}
\usepackage{hyperref}
\usepackage{tgpagella}

\usepackage{geometry}
\geometry{a4paper,left=15mm,right=15mm, bottom=15mm, top=15mm}

% empty page numbers
\pagestyle{empty}

\newlength{\spacebox}
\settowidth{\spacebox}{123456789}

\usepackage{sectsty}
\sectionfont{%                        
	\large % make sections smaller
	%\fontfamily{qag}\selectfont % change font family
	\sectionrule{0pt}{0pt}{-5pt}{1pt} % insert a thin rule
}

\newcommand{\sepspace}{\vspace*{1em}}


% Macro name : top of the CV
\newcommand{\name}[1]{
	\huge % font size
	%\fontfamily{phv}\selectfont % font family
	% print name centered and bold
	\begin{center} \textbf{#1} \end{center}\par
	% back to normal size and font
	\normalsize\normalfont}

\newcommand{\motto}[1]{
	\large % font size
	%\fontfamily{phv}\selectfont % font family
	% print motto centered and slanted
	\begin{center} \textsl{#1}\end{center}\par
	% back to normal size and font
	\normalsize \normalfont}


% personal information
\newcommand{\info}[2]{
	% set specific indentation for personal information
	\noindent\hangindent=2em\hangafter=0
	% create a box to align two pieces of text
	\parbox{60pt}{\textsl{#1}} % slanted entry name
		#2 \par
} % entry value


\newcommand{\skill}[2]{
	% set specific indentation for personal information
	\noindent\hangindent=2em\hangafter=0
	% create a box to align two pieces of text
	\parbox{3\spacebox}{% three times larger box
		\textsc{#1}} % small caps entry name
	#2 \par} % entry value
% language level
\newcommand{\lan}[2]{
	% set specific indentation for personal information
	\noindent\hangindent=2em\hangafter=0
	% create a box to align two pieces of text
	\parbox{\spacebox}{%
		\textbf{#1}} % bold font entry name
	#2 \par}    % entry value


\newcommand{\education}[4]{
	% name of the studies
	\noindent  \textbf{#1}
	% at the right the duration
	\hfill 
	\framebox{% duration inside a frame box
		\parbox{6em}{%
			\centering\textbf{#2}}} \par
	% new paragraph with the school in italics
	\noindent \textit{#3} \par
	% description with no hanging and in smaller text
	\vspace*{0.5em}
	\noindent\hangindent=2em\hangafter=0 \small #4 
	%back to normal size
	\normalsize \par}

\newcommand{\work}[4]{
	% name of the work
	\noindent  \textbf{#1}
	% at the right the duration
	\hfill 
	\framebox{% duration inside a frame box
		\parbox{6em}{%
			\centering\textbf{#2}}} \par
	% new paragraph with the school in italics
	\noindent \textit{#3} \par
	% description with no hanging and in smaller text
	\vspace*{0.5em}
	\noindent\hangindent=2em\hangafter=0 \small #4 
	%back to normal size
	\normalsize \par}

%\newenvironment{twenty}{%
%\begin{tabular*}{\textwidth}{@{\extracolsep{\fill}}ll}
%}{%
%\end{tabular*}
%}
%
%\newcommand{\twentyitem}[4]{%
%#1&\parbox[t]{0.88\textwidth}{%
%	\textbf{#2}%
%	\hfill%
%	{\footnotesize#3}\\%
%	#4\vspace{\parsep}%
%}\\
%}



\title{\Large {\bf Curriculum-Vitae} \\ Felipe FIGUEREDO ROCHA}
\date{}
\begin{document}
%\maketitle

\name{Felipe FIGUEREDO ROCHA}
\vspace*{-10pt}
\motto{Maître de conférences au Laboratoire de Modélisation et Simulation Multi-échelle (MSME) et \\ à la Faculté des Sciences et Technologie de l'Université Paris-Est Créteil Val de Marne (UPEC)}
% personal information

\info{Mise à jour}{27 Mai 2024}
\info{Nom complet}{Felipe FIGUEREDO ROCHA}
\info{Email}{\url{felipe.figueredo-rocha@u-pec.fr}, \url{f.rocha.felipe@gmail.com}}
\info{Orcid/Scholar}{\url{orcid.org/0000-0001-6893-1109}, \url{scholar.google.com/citations?user=mH05A4MAAAAJ&hl=en}
\info{Site/Github}{\url{felipefr.github.io/}, \url{github.com/felipefr}}
\info{Addresse}{Fac. des Sciences et Technologie, P2-336, 61, Av. Général de Gaulle, 94010 Créteil, France}
\info{Laboratoire}{Modélisation et Simulation Multi-échelle (MSME UMR 8208), équipe BIOMECA}
\info{Site Labo}{\url{msme.univ-gustave-eiffel.fr/}, \url{www.msme-upec.mooo.com/pagepro?name=Felipe\%20Rocha}}
}


\section*{Principaux mots-clés en recherche}
%I am currently postdoctoral fellow in the Applied Mathematics department at Ecole Polytechnique Fédéral de Lausanne since Dec'2019, working mainly with Reduced Models and Machine Learning applied for numerical solution of PDEs, under supervision of Prof. Annalisa Buffa and Prof. Simone Deparis. I earned my PhD and Master in Computing Modelling at National Laboratory for Scientific Computing (LNCC), Brazil, both under supervision of Prof. Pablo Blanco and Prof. Raúl Feijóo. In this period, I was also visitor student at the Zienkiewicz Centre for Computational Engineering, Swansea, United Kingdom, working with Prof. Eduardo de Souza Neto. My Bachelor background is in Mechanical Engineering, earned at Universidade Federal do Rio Grande do Norte (UFRN), Brazil, with an exchange period at École d'Arts et Métiers Paristech, France. My other main research interests are: constitutive modelling of soft tissues with emphasis in softening and failure mechanisms; computational homogenisation and multi-scale modelling; continuum mechanics with emphasis in nonlinear solid mechanics; numerical methods with emphasis in Finite Element Method.
%
%\leftline
Modélisation Multi-échelle; Principes variationnels pour la mécanique; Mécanique en grandes déformations; Tissus mous;  Matériaux fibreux; Mécanique pilotée par les donnés (\textit{Model-free}); Réseaux de neurones pilotées par la physique; Réduction de modéles; Eléments finis; Développement open-source (FEniCS).
\section*{Parcours académique}
\subsection*{Education}
\begin{description}
\item[Doctorat (2019)] DSc. Modélisation Numérique (PhD Computational Modelling)
\begin{itemize}
\item Institution: LNCC - Laboratoire National du Calcul Scientifique (National Laboratory for Scientific Computing),  Petropolis, Rio de Janeiro, Brésil.
\item Thèse: Multiscale Modelling of Fibrous Materials: from the elastic regime to failure detection in soft tissues (en anglais) (DOI \url{10.13140/RG.2.2.16031.28320}).
\item Directeurs: Prof. Pablo Javier Blanco et Prof. Raul Feijoo.
\item Echange international (Sep'2017-Fév'2018): avec Prof. Eduardo de Souza Neto au Zienkiewicz Centre for Computational Engineering (Swansea Universtity), Swansea, Pays de Gales, Royaume-Uni.
\item Moyenne: 3.83/4 
%\item \textbf{description:} 
%Mes études ont porté surtout sur l'utilisation du principe des puissances virtuels multi-échelle pour dériver des modèles en deux (ou plus) d'échelles dans les contextes physiques non-conventionnels, ici appliqué à de matériaux fibreux biologiques. Contrairement à des autres modèles souvent utilisés dans la littérature, on a adressé plusieurs difficultés qu'y sont inhérents, à savoir: i) le milieu continu macro-échelle et les fibres sont modélisées en déformation finies (non-linéarité géométrique); ii) les lois de comportement pour les fibres sont aussi non-linéaires (non-linéarité constitutive), en régime élastique pour bien représenter l'éffet du recrutement \footnote{En biomécanique, il s'agit du l'étirement et alignement progressif des chaînes des fibres, que ne sont pas raides initialement, de sorte que la raideur soit augmenté proportionnellement}, et l'endommagement pour bien modéliser la rupture de fibrilles lors d'un étirement important ; iii) les fibres ne sont pas attachés dans une matrice continue (un milieux fluide dans ce cas ), donc il ne s'agit pas d'un matériau renforcé par des fibres; iv) conditions aux limites non-affines, de façon à ne pas bloquer les modes de localisation de déformation. On remarque que les enjeux ci-mentionnés rendent la solution numérique associé au problème micro-échelle plus difficile, ce que donne lieu à des stratégies de régularisation visqueuse consistent (voir Section \ref{sec:travaux} pour les références).
\end{itemize}
\item[Master (2015)] MSc. Modélisation Numérique (Master in Computational Modelling)
\begin{itemize}
\item Institution: LNCC - Laboratoire National du Calcul Scientifique (National Laboratory for Scientific Computing),  Petropolis, Rio de Janeiro, Brésil.
\item Thèse: Basics Aspects of Multi-Scale Modelling of Biological Tissues (en portugais) (DOI \url{10.13140/RG.2.2.15484.41603}).
\item Directeurs: Prof. Pablo Javier Blanco et Prof. Raul Feijoo.
\item Moyenne: 3.90/4 
\end{itemize}

\item[Bachelor (2013) (Summa cum Laude)] Génie Mécanique (B.E. Mechanical Engineering)
\begin{itemize}
	\item Institution: UFRN - Université fédérale du Rio Grande do Norte (Federal University of Rio Grande do Norte),  Natal, Rio grande do Norte, Brésil.
	\item Echange international (Aug'2011-Jui'2012): Expertise \textit{Prototypage Virtuel} au 3éme année à Arts et Métiers Paristech (ENSAM), Paris, France.
	\item Mémoire 1: Development of a Computational Dynamics Software for Multiple Rigid Bodies Analysis (en portugais) - encadré par Prof. Wallace Bessa.
	\item Mémoire 2: Modélisation et simulation numérique de l'usinage d'une pièce automobile - encadré par Prof. Philippe Lorong and Prof. Jerôme Duchemin.
	\item Moyenne: 8.85/10
\end{itemize}
\end{description}

\subsection*{Stages post-doctoraux}
\begin{description}
\item[Post-doctorant (2022-2023, 18 mois) avec Prof. Laurent STAINIER].
\\
\textbf{Laboratoire/Université:} GeM (UMR 6183)/École Centrale de Nantes, Nantes, France.\\
\textbf{Activités:} Développement de la méthode \emph{(Model-free) Data-driven Computational Mechanics} en plusieurs volets: en dialogue avec l'homogéisisation numérique, comparaisons avec les réseaux de neurones, pour les transformations finis. Chargé de TDs et TPs aux cours de Mécanique de Milieux Continus en première année d'école d'ingénieurs et première année du master international \emph{Computational Mechanics}. Encadrement de projets de fin d'étude, doctorant visitant.
\item[Post-doctorant (2019-2021, 24 mois) - avec Prof. Annalisa BUFFA et Prof. Simone DEPARIS] . \\
\textbf{Laboratoire/Université:} Chair of Numerical Modelling and Simulation/Ecole polytechnique fédérale de Lausanne, Suisse. \\
\textbf{Activités:}
Proposition d'une condition aux limites pilotée par les données (DeepBND) pour l'homogénéisation numérique. Développement des librairies \emph{open-sources} libres \textit{DeepBND} et \textit{micmacsfenics}. Assistant d'enseignement dans la section de  Mathématiques.   
\end{description}

\subsection*{Visites académiques de courte durée}

\begin{description}
\item[Sep-Out'19 (1,5 mois)] \textbf{Prof. Sidarta Lima (UFRN, Natal, Brésil)}: Cours en formulations mixtes pour l'équation de Darcy. Implementation des éléments de Raviart-Thomas dans un code in-house. Visite menée dans le cadre d'un projet avec Petrobras (entreprise d'État brésilienne du secteur d'énergetique).
\item[Avr'17 (3 semaines)] \textbf{Prof. Anne Robertson (University of Pittsburgh, Pittsburgh, USA)}: J'ai participé aux essais mécaniques et d'imagérie microscopique dans les artères de moutons pour mieux comprendre les aspects microscopiques des fibres de collagène aux tissus mous.
\item[Jui'-Jul'15 (1 mois)] \textbf{Prof. Pablo Sanchez et Prof. Alfredo Huespe (Centro de Investigación de Métodos Computacionales (CIMEC), Santa Fe, Argentina)}{ Etude sur des aspects théoriques et numériques en endommagement et fissuration de matériaux: régularisation des lois d'endommagement, détection de bifurcation par l'analyse spectrale du tenseur acoustique.}
	
\end{description}



%\twentyitem{Aug-Jul'12}{Arts et Métiers Paristech (ENSAM), Paris, France}{}{I took part of undergraduate studies in a sandwich exchange program in France. I was enrolled in the third (last) year, which consisted in a  masters degree in ''Prototypage Virtuel'' (Computational Solid Mechanics) including a final research project. (Funded by  Brafitec/CAPES).}
%
%\twentyitem{Sep-Fev'18}{Prof. Eduardo de Souza Neto, Swansea University, Swansea, UK}{}{I visited the Zienkiewicz Centre for Computational Engineering (ZCCE) as part of the sandwich exchange program during my PhD. During this period, under supervision of Prof. de Souza Neto, expert in the domain of Computational Plasticity and Constitutive Multiscale theories, we developed and implemented a nouvel micromechanical inelastic model for networks of fibres. I was funded by PSDE/CAPES.}

\section*{Production scientifique} \label{sec:travaux}

\subsection*{Articles publiés en revues}
\begin{enumerate}
	\item Pablo Javier Blanco, Pablo Javier Sánchez, Felipe Figueredo Rocha, Sebastian Toro, and Raúl Antonino Feijóo. A consistent multiscale mechanical formulation for media with randomly distributed voids. \textit{International Journal of Solids and Structures}, 2023.
	\item Felipe Figueredo Rocha, Simone Deparis, Pablo Antolin, Annalisa Buffa.  Deepbnd : A machine learning approach to enhance multiscale solid mechanics, \textit{Journal of Computational Physics}, 2023.
	\item Felipe Figueredo Rocha, Pablo Javier Blanco, Pablo Javier Sánchez, Eduardo de Souza Neto, and Raúl Antonino Feijóo. Damage-driven strain localisation in networks of fibres: A computational homogenisation approach. \textit{Computer \& Structures}, 2021.
	\item Felipe Figueredo Rocha, Pablo Javier Blanco, Pablo Javier Sánchez, and Raúl Antonino Feijóo. Multi-scale modelling of arterial tissue: Linking networks of fibres to continua. \textit{Computer Methods in Applied Mechanics and Engineering}, 2018
\end{enumerate}	
\subsection*{Articles submis/en preparation en revues}
\begin{enumerate}
\item Martin Zlatic, Felipe Rocha, Marko Canidija, Laurent Stainier. Data-driven methods for computational mechanics : a fair comparison between neural networks and model-free approches  (submitted), \textit{Computer Methods in Applied Mechanics and Engineering}, 2024.
\item Felipe Rocha, Auriane Platzer, Adrien Leygue, Laurent Stainier. An active learning approach for (Model-Free) Data-driven mechanics using computational homogenisation  (in preparation), \textit{Mechanics of Materials}, 2024.
%\item Kim Quang Hoang, Felipe Figueredo Rocha, Eduardo de Souza Neto. An
%RVE-based multiscale approach to the micro-discrete to
%macro-continuum transition in atomistic modelling:
%application to graphene and boron-nitride modelling (in preparation). \textit{journal of Mechanics and Physics of Solids}, 2023.
\end{enumerate}
\subsection*{Articles complets en conferences} 
\begin{enumerate}
	\item Felipe Rocha, Thiago Quinelato, Laurent Stainier. Some experiences in mixed finite element formulations for (model-free) data-driven computational mechanics, \textit{16$^{e}$ Colloque National en Calcul des Structures}, Giens, France.
	\item Felipe Rocha, Simone Deparis, Pablo Antolin, Annalisa Buffa. 
	A divide-to-conquer approach to a hybrid ROM-NN method for multi-scale problems: the robustness assessment for incomplete information scenarios, \textit{Congrès Français de Mécanique}, Nantes, France, 2022.
	\item F.F. Rocha, P.J. Blanco, R.A. Feijóo, P.J. Sanchez, and A.E. Huespe. A multi-scale approach to model arterial tissue. \textit{In Ibero-Latin American Congress on Computational Methods in Engineering} (CILAMCE), Rio
	de Janeiro, Brazil, 2015.
\end{enumerate}	
\subsection*{Résumés étendues en conferences}
\begin{enumerate}
	\item F.F. Rocha; P.J. Blanco ; P.J. Sánchez; R.A. Feijóo. On the constitutive modeling for fibrous tissues. In: International Conference on Computational and Mathematical Biomedical Engineering, 2017, PITTSBURGH. International Conference on Computational and Mathematical Biomedical Engineering Proceedings, 2017.
\end{enumerate}
\subsection*{Résumés et presentations en conferences} 
\begin{enumerate}
	\item Felipe Rocha, Laurent Stainier. ddfenics: a FEniCS-based (Model-Free)
	Data-driven Computational Mechanics implementation, \textit{FEniCS 2023 Conference}, 14-16 June, Cagliari, Italy \url{https://fenicsproject.org/assets/extra/fenics2023/FEniCS2023_program.pdf}.
	\item Felipe Rocha, Auriane Platzer, Andrien Leygue, Laurent Stainier, Michael Ortiz, A model-free data-driven paradigm for multi-scale mechanics, IUTAM Symposium on Data-Driven Mechanics and Surrogate Modeling, Arts et Métiers Institute of Technology, October 25-28, 2022 \url{https://iutamddmech.i3a.es/}.
	\item Felipe Rocha, Auriane Platzer, Andrien Leygue, Laurent Stainier, Michael Ortiz, A Model-free Data-driven Approach for Computational Homogenisation, 9th GACM Colloquium on Computational Mechanics 2022, 21 - 23 Essen, Germany \url{https://colloquia.gacm.de/organisation}.
	\item Felipe Figueredo Rocha, Simone Deparis, Pablo Antolin, Annalisa Buffa, 
	DeepBND: Using a hybrid ROM-NN approach to accelerate Computational Homogenisation in Solid Mechanics, 8th ECCOMAS 2022 - European Congress on Computational Methods
	in Applied Sciences and Engineering, Oslo, Norway, 5th - 9th of June, 2022 \url{https://www.eccomas2022.org/frontal/default.asp}.
	\item Felipe Figueredo Rocha, Simone Deparis, Pablo Antolin, Annalisa Buffa, 
	DeepBND: a Machine Learning approach to enhance Multiscale Solid Mechanics, 18th European Mechanics of Materials Conference (EMMC18)
	April 4 - 6, 2022, Oxford, UK.
	\item F.F. Rocha; P.J. Blanco; de Souza Neto, E.; P.J. Sánchez, R.A. Feijóo. An computational homogenisation approach to assess the strain localisation due to damage in fibre networks. XVI International Conference on Computational Plasticity. Fundamentals and Applications, Barcelona, Spain, 2021, CIMNE.
	\item F.F. Rocha; P.J. Blanco; de Souza Neto, E.; P.J. Sánchez, R.A. Feijóo. An computational homogenisation approach to assess the strain localisation due to damage in fibre networks. XVI International Conference on Computational Plasticity. Fundamentals and Applications, Barcelona, Spain, 2021, CIMNE.
	\item P.J. Blanco, P.J. Sánchez, F.F. Rocha, Toro, S.; R.A. Feijóo.
	Multiscale formulation for materials with randomly distributed voids: minimally constrained and more restrictive multiscale sub-models. In: XII Argentine Congress on
	Computational Mechanics, 2018, San Miguel de Tucumán.
	Mecánica Computacional. Santa Fé: Asociación Argentina de Mecánica Computacional, 2018. v.XXXVI.
	p.1683 - 1683
	\item F.F. Rocha; P.J. Blanco; de Souza Neto, E.; P.J. Sánchez, R.A. Feijóo.
	Towards post-critical multiscale modelling of damage in biological fibrous tissues.
	In: XII Argentine Congress on Computational Mechanics, 2018, San Miguel de Tucumán.
	Mecánica Computacional. Santa Fé: Asociación Argentina de Mecánica Computacional, 2018. v.XXXVI.
	p.1875 - 1875
	\item F.F. Rocha, P.J. Blanco, P.J. Sánchez, R.A. Feijóo.
	A Multiscale Approach to Study Softening Mechanisms in Arterial Tissue In: EMI2017-IC - 2017 EMI
	International Conference, 2017, Rio de Janeiro. EMI2017-IC - 2017 EMI International Conference Proceedings. , 2017.
	\item Toro, S., F.F. Rocha, P.J. Sánchez, P.J. Blanco, A.E. Huespe, R.A. Feijóo.
	Modelado Multiescala de Materiales: Análisis de Condiciones de Borde en Micro-Estructuras con Poros y/o
	Inclusiones que Alcanzan la Frontera del RVE In: Congreso sobre Métodos Numéricos y sus Aplicaciones,
	2017, La Plata. Anais do ENIEF 2017. La Plata: Asociación Argentina de Mecánica Computacional, 2017. v.XXXV. p.1309 -1309.
\end{enumerate}


\subsection*{Seminaires invités}
\begin{enumerate}
	\item \textbf{(Mar'2023) XV Simpósio de Análise Numérica e Otimização (UFPR, Brazil) (visio)}  Data-driven/Machine-learning approaches for computational homogenisation: replacing classical boundary conditions and constitutive models by data
	\item \textbf{(Nov'2021) CRUNCH Group: Machine Learning + X seminars, Brown University (visio, disponible en ligne)}  DeepBND: a Machine Learning approach to enhance Multiscale Solid Mechanics
	\item \textbf{EAMC 2021 (2021), LNCC} Galerkin convida Mr. Deep para um café (in Portuguese).
	\item \textbf{Alumni Post-graduate Seminar (2021), LNCC} Aprendizado de Máquina em Computação Científica com Aplicações à Solução Numérica de EDPs (in Portuguese).
\end{enumerate}

\subsection*{Développement \emph{open-source}} \label{sec:libraries}
\begin{enumerate}
	\item \textbf{(2022-) ddfenics:} a FEniCs-based (Model-Free) Data-driven Computational Mechanics implementation \url{zenodo.org/badge/latestdoi/545056382}.
	\item \textbf{(2020-) micmacsfenics:} micmacsfenics: a FEniCs-based implementation of two-level finite element simulations (FE2) using computational homogenization. \url{zenodo.org/badge/latestdoi/341954015}.
	\item \textbf{(2020-) fetricks:} Useful tricks and some extensions for FEniCs and other FEM-related utilities (FE + tricks : where FE stands for Fenics and Finite Element) \url{zenodo.org/badge/latestdoi/489339019}
	\item \textbf{(2020-2022) deepbnd:} implementation of the DeepBND method based on FEniCs and Tensorflow \url{zenodo.org/badge/latestdoi/296098609}
%	\item \textbf{(2023-) matlib-fenics:} a FEniCs wrapper for Matlib (a general-purpose library for material behaviours) - avec Laurent Stanier (pas encore opensource).
\end{enumerate}

\section*{Enseignements}
\begin{description}
\item[(2023-) à la FST-UPEC]:
\begin{itemize}
\item Mécanique du Point 1, L1-Chimie et Maths, TD $2\times 16,5$ h, 2023.2.  
\item Mécanique du Point 2, L1-Chimie et SPI, TD $2\times 16,5$ h + TP $2\times3$ h,  2024.1.
\item Résistance des matériaux, L1-SPI, TD $3\times 12$ h + TP $2\times6$ h,  2024.1. 
\item Analyse Numérique et Calcul Scientifique, M1-MMSOL, CM $4,5$h + TP $12$h.  
\item Simulation Numérique en Mécanique des Solides, M1-MMSOL, TP $18$h.
\end{itemize}
\item[(2022-2023) à l'Ecole Centrale de Nantes]:
\begin{itemize}
	\item Modélisation et mécanique des milieux continus, L3 Ingénieur généraliste, TP 24h, 2023.1.
	\item Continuum Mechanics, M1 Computational Mechanics, TD 12h, 2022.2.  
	\item Modélisation et mécanique des milieux continus, L3 Ingénieur généraliste, TP 24h, 2022.1.
\end{itemize}
\item[(2020-2021) à l'EPFL]:
\begin{itemize}
\item Analyse II, L1 Génie Matériaux, Microtechnique, TD 28h, 2021.1.
\item Analyse Numérique, L2 Sciences de la vie, TP 28h, 2020.1.  
\end{itemize}
\item[(2018) au LNCC]:
\begin{itemize}
\item Méthodes numériques (Portugais), Master - Computational Modelling, TD 12h, 2018.2.  
\item Fondements de la modélisation (Portugais), Master - Computational Modelling, TD 12h, 2018.1.
\end{itemize}
\end{description}

\section*{Encadrements}
\begin{itemize} 
\item \textbf{Avr 2024 - }, Dona Elisa BOU ZEIDAN, \textit{Prédiction des propriétés mécaniques efectives des microstructures osseuses par l'analyse d'images et méthodes d’apprentissage automatique}, post-doc, allocations FST-UPEC, \textit{co-encadré par Vittorio SANSALONE (MSME) et  Amine NAIT-ALI (LISSI)}. 
\item \textbf{Nov 2023 - Mars 2024}, Mustapha HAMRIOU et Anis ZEMRI, \textit{Prédiction du tenseur élastique des matériaux composites par l’apprentissage automatique : une preuve de concept}, Projet final Master 2 parcours Modélisation et Simulation en Mécanique des Solides, UPEC - \textit{co-encadré par Vittorio SANSALONE (MSME)}.
\item \textbf{Nov 2022 - Mars 2023}, Clément AUBORG et Margaux DELAGE, \textit{Implémentation d’un solveur DDCM (Data-Driven Computational Mechanics) en grandes déformations}, Projet final option Modélisation Avancée et Analyse des Structures (MAAS), Ecole Centrale de Nantes.
\end{itemize}

\section*{D'autres parcipation aux conférénces (sans presentation)}
\begin{itemize}
\item \textbf{(12 Fév 2024)}, Demi-journée GT Grandes Transformations, ENS Paris-Saclay, Gif-sur-Yvette.  
\item \textbf{(14 Mars 2024)}, Workshop GdR I-GAIA / CNRS\@ CREATE, Physics Informed Learning, ENSAM, Paris.
\item \textbf{(27 – 29 Nov 2023)}, GdR I-GAIA - GdR Week 2023, ENSAM, Paris. 
\item \textbf{(16 -18 Oct 2023)}, The 57th Meeting of Society for Natural Philophy, \textit{Complex Material Structures, Natural and Architected}, Paris. \url{http://snp.uniud.it}.
\item \textbf{(16-20 Mai 2022)}, 15$^{e}$ Colloque National en Calcul des Structures, Giens, France.
\item \textbf{(14-15 Mai 2022)}, Juniors 15$^{e}$ Colloque National en Calcul des Structures, Giens, France.
\item \textbf{(10 Mars 2022)} Journée scientifique Matériau-Numérique, SF2M - Société  Française de Métallurgie et Matériaux, Paris, France.
\item \textbf{(13 Septembre 2021} Swiss Numerics Day 2021, Lausanne, Suisse  \url{https://snd2021.epfl.ch/}. 
\item \textbf{(8-10 Jan 2020)} International Workshop on Scientific Machine Learning, University of Cologne, Germany.
\item \textbf{(20-24 Jul 2020)} MSML2020 Mathematical \& Scientific Machine Learning Conference, Online, Princeton University (Online Event).
\item \textbf{(3-7 Aug 2020)} MATHML2020,  LMS-Bath Symposium on the Mathematics of Machine Learning, Online, University of Bath (Online Event).
\end{itemize}



\section*{Education complementaire}
\begin{itemize}
\item \textbf{2022 (30h)} DATA-DRIVEN MECHANICS:
CONSTITUTIVE MODEL-FREE APPROACH, CISM, Udine, Italy.
\item \textbf{2020 (16h, listener)} (MATH-631) Mathematical foundations of neural networks, EPFL, Lausanne, Switzerland.
\item \textbf{2019 (36h, listener)} Introdução ao Aprendizado de Máquina, LNCC, Petrpolis, Brazil.
\item \textbf{2018 (4h)} Python for HPC, LNCC, Petropolis, Brazil.
\item \textbf{2015 (6h) } New Formulations of Finite Element Method, LNCC, Petropolis, Brazil.
\item \textbf{2014 (32h) } Biomech. Summerschool: Trends of Modelling, TUGraz, Graz, Austria.
\item \textbf{2014 (4h) } Topological Asymptotic Analysis, LNCC, Petropolis-RJ, Brazil. 
\item \textbf{2014 (7h) } Object-Oriented Finite Element Method, LNCC, Petropolis-RJ, Brazil.
\end{itemize}

\section*{Reviewer} 
\begin{itemize}
\item \textbf{2022-} European Journal of Mechanics - A/Solids, Elsevier.
\item \textbf{2022-} Computational and Applied Mathematics, Springer.
\item \textbf{2020-} Engineering Computations, Emerald Publishing.
\item \textbf{2021} National Congress of Applied and Computational Mathematics, CNMAC 2021 (Online Event), Brasil. 
\item \textbf{2021}  Academic Meeting on Comp. Modelling, EAMC 2021, LNCC, Petropolis, Brazil. \\

\end{itemize}

\section*{Participation en jury} 
\begin{itemize}
\item \textbf{2022, Bachelor's Final Project} Vinícius Jucá Policarpo, Desenvolvimento de uma ferramenta educacional para o calculo de projeto de fadiga em aços (en Français: "développement d’un outil pédagogique pour le calcul et projet en fatigue pour des aciers", UFC, Fortaleza-CE, Brazil.
\item \textbf{2020, Bachelor's Final Project} Henrique Ribeiro da Silva, Dinâmica dos Fluidos Computacional: Uma aproximação paralelizada via Smoothed Particle Hydrodynamics, CEFET, Petropolis-RJ, Brazil.
\end{itemize}


\section*{Organisation d'événéments}
\begin{itemize}
\item \textbf{2024} Méca-J 2024, Congrès des Jeunes Chercheurs en Mécanique, en ligne, France - \textit{mis en place du site-web, comité scientifique} \url{https://mecaj2024.sciencesconf.org/}.
\item \textbf{2022} Congrés Français de Mécanique, Nantes, France - \textit{organisation des salles}.
\item \textbf{2017} Festival des mathématiques, Rio de Janeiro, Brésil - \textit{organisation des salles}.
\end{itemize}

\section*{Autres Compéténces}
\textit{Note: F-Fluent, M-Moyen, B-Basique}. 
\begin{itemize}
\item \textbf{Linguitiques}: Portuguese (F), Français (F), Anglais (F), Italien (M), Espagnol (M).
\item \textbf{Programmation}: Python (F), C/C++ (F), Fortran 77/90 (F), Matlab (B).
\item \textbf{Libraries}: FEniCS (F), Tensorflow/Keras (F), Numpy/Scipy (F), OpenMP (B), MPI/mpi4py (B), Petsc (B). 
\item \textbf{Outils}: Linux/Mac (F), Latex (F), Gmsh (F), , Git (M), Bash (M). 
\item \textbf{Logiciels}: Paraview (F), Inkscape (F), Office (F), COMSOL (B).
\end{itemize}

\section*{D'autres prix}
\begin{itemize}
\item \textbf{2024 - Projet Soutien Recherche JC 2024}: concernent aux nouveaux MCFs de la FST-UPEC. Investissement matériel à l'hauteur de 6K\texteuro.
\item \textbf{2021 - Grant INRIA projet fleché Brésil (pas pris)} : projet concernant PINNs pour l'électromagnetism à l'équipe ATLANTIS (Sophia-Antipolis).
\item \textbf{2019 - Concours pour l'accées à la fonction publique (pas pris)} Professor assistente (Maître de Conférences), Mathématiques appliqués, Université fédérale de Bahia, Brésil.
\item \textbf{2017 - ''Aluno nota 10'' (LNCC):} bourse FAPERJ, meilleur étudiant du programme doctorale.
\item \textbf{2014 - ''Aluno nota 10'' (LNCC):} bourse FAPERJ, meilleur étudiant du programme de master.
\item \textbf{2012 - Summa cum Laude (UFRN): } Licence en Génie Mécanique.
\item \textbf{2006 - Medaille d'argent OBMEP-IMPA:} Olympiades Brésiliennes en Mathématique des écoles publiques.
\end{itemize}

%\section*{Références}
%\par \textit{Note 1: Postes écrits en anglais quand la correspondance en français n'est pas nette.} 
%\par \textit{Note 2: Classés en ordre chronologique.} 
%\begin{description}
%\item[Laurent Stainier] Professeur des universités (ECN),
%\par \info{relation}{Encadrant post-doctorat. Responsable de cours dont j'ai été assistant.}
%\par \info{contact}{\url{laurent.stainier@ec-nantes.fr}.}
%\item[Thomas Heuzé] Maître des Conférences HDR (ECN),
%\par \info{relation}{Responsable de cours dont j'ai été assistant.}
%\par \info{contact}{\url{thomas.heuze@ec-nantes.fr}.}
%\item[Annalisa Buffa] Full Professor (EPFL), Directrice de la Chaire “Numerical Modelling and Simulation”, \par \info{relation}{Encadrant post-doctorat.}
%\par \info{contact}{\url{annalisa.buffa@epfl.ch}.}
%\item[Simone Depais] Adjunct Professor (EPFL), Directeur exécutif du centre propédeutique , 
%\par \info{relation}{Encadrant post-doctorat. Responsable de cours dont j'ai été assistant.}
%\par \info{contact}{\url{simone.deparis@epfl.ch}}
%\item[Pablo Javier Blanco] Full Researcher (LNCC),  head of HeMoLab group (\url{http://hemolab.lncc.br/}), co-PI of the National Institute of Science and Technology in Medicine Assisted by Scientific Computing (INCT-MACC, \url{https://macc.lncc.br}).
%\par \info{relation}{Directeur et encadrant de thèse doctorat et master.}
%\par \info{contacts}{\url{pjblanco@lncc.br}}
%\item[Eduardo de Souza Neto] Full Professor (Swansea University),  head of civl engineering department.
%\par \info{relation}{encadrant pendant le période de mobilité à l'étranger lors de la thèse, collaborateur.}
%\par \info{contacts}{\url{e.desouzaneto@swansea.ac.uk}}
%\end{description}



%\section*{Références Complete}
%\par \textit{Note 1: Postes écrits en anglais quand la correspondance en français n'est pas nette.} 
%\par \textit{Note 2: Classés en ordre chronologique.} 
%\begin{description}
%\item[Laurent Stainier] Professeur des universités (ECN),
%\par \info{relation}{Encadrant post-doctorat. Responsable de cours dont j'ai été assistant.}
%\par \info{contact}{\url{laurent.stainier@ec-nantes.fr}, 02 40 37 25 86.}
%\item[Thomas Heuzé] Maître des Conférences HDR (ECN),
%\par \info{relation}{Responsable de cours dont j'ai été assistant.}
%\par \info{contact}{\url{thomas.heuze@ec-nantes.fr}, 02 40 37 25 03.}
%\item[Annalisa Buffa] Full Professor (EPFL), Directrice de la Chaire “Numerical Modelling and Simulation”, \par \info{relation}{Encadrant post-doctorat.}
%\par \info{contact}{\url{annalisa.buffa@epfl.ch}, +41 21 693 14 30.}
%\item[Simone Depais] Adjunct Professor (EPFL), Directeur exécutif du centre propédeutique , 
%\par \info{relation}{Encadrant post-doctorat. Responsable de cours dont j'ai été assistant.}
%\par \info{contact}{\url{simone.deparis@epfl.ch}, +41 21 693 25 47.}
%\item[Pablo Javier Blanco] Full Researcher (LNCC),  head of HeMoLab group (\url{http://hemolab.lncc.br/}), co-PI of the National Institute of Science and Technology in Medicine Assisted by Scientific Computing (INCT-MACC, \url{https://macc.lncc.br}).
%\par \info{relation}{Directeur et encadrant de thèse doctorat et master.}
%\par \info{contacts}{\url{pjblanco@lncc.br}, +55 24 22336067.}
%
%\item[Eduardo de Souza Neto] Full Professor (Swansea University),  head of civl engineering department.
%\par \info{relation}{encadrant pendant le période de mobilité à l'étranger lors de la thèse, collaborateur.}
%\par \info{contacts}{\url{e.desouzaneto@swansea.ac.uk}, +44 (0) 1792 295256.}
%\end{description}

\end{document}
